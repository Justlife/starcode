\documentclass[12pt]{article}
\usepackage{url}
\title{Starcode tutorial}
\author{Guillaume Filion}

\begin{document}
\maketitle

\section{Build instructions}

Starcode runs on Mac and Linux as a command line application.
You can download the source code of Starcode from Github with
the following command on a standard terminal.

\begin{verbatim}
git clone git@github.com:gui11aume/starcode
\end{verbatim}

Note that this requires that you already have a Github account
and that the computer you are working on has an SSH  key registered
on Github. If this is not the case, follow the instructions from
\url{https://help.github.com/articles/generating-ssh-keys/}.

This should download a directory named \texttt{starcode}. To build
Starcode, execute the following.

\begin{verbatim}
cd starcode
make
\end{verbatim}

This should succeed on most Linux systems because \texttt{make} is
available by default. Should it not be the case, you can obtain it by
typing \texttt{sudo apt-get install make} on Ubuntu. On Mac, you need
to install XCode, which may take some time. First, you will need an
Apple ID, then you will need to download it from the developer
website of Apple \url{https://developer.apple.com/xcode/downloads/}.
Then, you may need to follow the instructions shown on the following
link to install the command line version of \texttt{make}
\url{http://stackoverflow.com/q/10265742/1248687}.

This should have created an executable called \texttt{starcode}. To
check that the building is successful, execute the following command
in the same directory.

\begin{verbatim}
./starcode test/test_file.txt
\end{verbatim}

You should obtain the following output.

\begin{verbatim}
running starcode with 1 thread
reading input files
raw format detected
sorting
setting dist to 2
progress: 100.00%
AGGGCTTACAAGTATAGGCC  7
CCTCATTATTTGTCGCAATG  7
GGGAGCCCACAGTAAGCGAA  7
TAGCCTGGTGCGACTGTCAT  7
TGCGCCAAGTACGATTTCCG  7
\end{verbatim}

\section{Starcode basics}

The previous example is a standard invocation of Starcode on sequence
data. The input file \texttt{test/test\_file.txt} is in raw format
(\textit{i.e.} plain text), it contains 35 lines, each with one
sequence of 20 nucleotides. By default, Starcode is verbose and prints
some information about the clustering process. In the example above,
Starcode used a single CPU, and it clustered the sequences within
edit distance 2 of each other.

The output starts after the line \texttt{progress: 100.00\%}. It
consists of five lines (one per cluster) where the sequence is the
centroid of the cluster (the most representative sequence) and the
number is its size (the number of sequences in the cluster).

We can make Starcode quiet (non verbose) with the \texttt{-q} option,
and we can set the clustering distance to 0 with \texttt{-d0}.

\begin{verbatim}
./starcode -q -d0 test/test_file.txt
\end{verbatim}

The output is the following.

\begin{verbatim}
AGGGCTTACAAGTATAGGCC  6
CCTCATTATTTGTCGCAATG  6
GGGAGCCCACAGTAAGCGAA  6
TAGCCTGGTGCGACTGTCAT  6
TGCGCCAAGTACGATTTCCG  6
AGGGGTTACAAGTCTAGGCC  1
CCTCATTATTTACCGCAATG  1
GGAAGCCCACAGCAAGCGAA  1
TAACCTGGTGCGACTGTTAT  1
TGCGCCAAGTAAGAATTCCG  1
\end{verbatim}

This time we obtain five clusters of size 6 and five clusters of size 1.
Each centroid of the clusters of size 1 has 2 mismatches with one
of the centroids of the clusters of size 6, which is why they form
separate clusters when the distance is less than 2.

As a side note, this example output illustrates that the clusters
are sorted first by size and then by alphabetical order of the
centroid.

\end{document}
